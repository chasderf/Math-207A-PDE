\documentclass{article}
\usepackage{amsmath}
\usepackage[mathscr]{euscript}
\newcommand\tab[1][1cm]{\hspace*{#1}}
\begin{document}


\title {Math 207A Ordinary Differential Equations: Ch.5 Lyapunov's Theorems and Methods}

\author{Charlie Seager}

\maketitle

\textbf{Lyapunov's Theorems}

\textbf {Lemma 5.1} If $\mathscr{n}$ is any point of D that is not a critical point of (5.6), then through the point $\mathcal{n}$ there passes at most one orbit of (5.6)

\textbf {Lemma 5.2} If any orbit C of (5.6) passes through some ordinary point of D, then C cannot reach any critical point $\alpha$ in D in finite time. (More precisely, if C is generated by a solution $\phi$ and if $\lim_{t \to a} \phi(t) = \alpha, \alpha$ in D, then $a = +- \infty$).

\textbf {Lemma 5.3} An orbit C of (5.6) that passes through at least one ordinary point of D cannot cross itself, unless, it is a closed curve in D. In this case, C corresponds to a periodic solution of (5.6)

\textbf {Definition 1} The scalar function V(y) is said to be positive definite on the set $\Omega$ if and only if V(0) = 0 and $V(y) > 0$ for $y \neq 0$ and y in $\Omega$.

\textbf {Definition 2} The scalar function V(y) is negative definite on the set $\Omega$ if and only if -V(y) is positive definite on $\Omega$.

\textbf {Definition 3} The derivative of V with respect to the system y'=f(y) is the scalar product
\begin{center}
$V*(y) = grad V(y) \cdot f(y) = \frac{\partial V}{\partial y_1} (y) f_1 (y) + \frac{\partial V}{\partial y_2} (y) f_2(y) + \dots + \frac{\partial V}{\partial y_n} (y)f_n(y)$
\end{center}

\textbf {Theorem 5.1} If there exists a scalar function V(y) that is positive definite and for which $V*(y) \leq 0$ (that is, the derivative (5.7) with respect to y'=f(y) is nonpositive) on some region $\Omega$ containing the origin, then the zero solution of y'=f(y) is stable.

\textbf {Theorem 5.2} If there exists a scalar function V(y) that is positive definite and for which V*(y) is negative definite on some region $\Omega$ containing the origin, then the zero solution of y'=f(y) is asymptotically stable.

\textbf {Theorem 5.3} If there exists a scalar function V(y), V(0) = 0, such that V*(y) is either positive definite or negative definite on some region $\Omega$ containing the origin and if there exists in every neighborhood N of the origin, $N \subset \Omega$ at least one point $a \neq 0$ such that V(a) has the same sign as V*(y), then the zero solution of y'=f(y) is unstable.

\textbf {Theorem 5.4} If there exists a scalar function V such that in a region $\Omega$ containing the origin, 
\begin{center}
$V* = \lambda V + W$
\end{center}
where $\lambda > 0$ is a constant and W is either identically zero or W is a nonnegative or a nonpositive function such that in every neighborhood N of the origin, $N \subset \Omega$, there is at least one point a such that $V(a) \cdot W(a) > 0$, then the zero solution of y'=f(y) is unstable.

\textbf {Chapter 5.3 Proofs of Lyapunov's Theorems}

\textbf {Chapter 5.4 Invariant Sets and Stability}

\textbf {Definition 1} A set $\Gamma$ of points in $E_n$ is (positively) invariant with respect to the system (5.6) if every solution of (5.6) starting in $\Gamma$ remains in $\Gamma$ for all furture time*.

\textbf {Definition 2} A point p in $E_n$ is said to lie in the positive limit set L(C*) (or is said to be a limit point of the orbit C*) of the solution $\phi(t)$ if and only if there exists a sequence $\{t_n\} \to + \infty$ as $n \to \infty$ such that $\lim_{n \to \infty} \phi(t_n) = p$.

\textbf {Lemma 5.4} If the solution $\phi(t,y_0)$ is bounded for $0 \leq t < \infty$ (that is, if there exists a constant M such that $||\phi(t, y_0)|| \leq < M$ for $0 \leq t < \infty$) then its positive limit set $L(C^+)$ is a nonempty invariant set (with respect to (5.6)). Moreover, the solution $\phi(t,y_0)$ approaches the set $L(C^+)$ as $t \to + \infty$ (in the sense that for each $\epsilon > 0$ there exists a $T > 0$ such that for every $t > T$ there exists a point p in $L(C^+)$ (possibly depending on t) such that $||\phi(t,y_0) - p|| < \epsilon$; that is, for t sufficiently large the semiorbit of the solution $\phi(t,y_0)$ lies arbitrarily close to points of $L(C^+))$.

\textbf {Lemma 5.5} Let V be continuously differentiable in a set $\Omega$ containing the origin and let $V*(y) \leq 0$ at all points of $\Omega$. Let $y_0 \in \Omega$ and let $\phi(t, y_0)$ be a bounded solution of (5.6) whose positive semiorbit $C^+$ lies in $\Omega$ for all $t \geq 0$ and let the positive limit set $L(C^+)$ of $\phi(t, y_0)$ lie in $\Omega$. Then V*(y) = 0 at all points of $L(C^+)$.

\textbf {Theorem 5.5} Let V(y) be a nonnegative scalar function defined on some set $\Omega \subset R_n$ containing the origin. Let V be continuously differentiable on $\Omega$, let $V*(y) \leq 0$ at all points of $\Omega$, and let V(0) = 0. For some real constant $\lambda \geq 0$ let $C_\lambda$ be the component of the set $S_\lambda = \{y | V(y) \leq \lambda \}$ which contains the origin. Suppose that $C_\lambda$ is a closed bounded subset of $\Omega$. Let E be the subset of $\Omega$ defined by $E = \{ y | V*(y) = 0\}$. Let M be the largest positively invariant subset of E (with respect to (5.6)). Then every solution of (5.6) starting in $C_\lambda$ at t = 0 approaches the set M as $t \to + \infty$

\textbf {Chapter 5.5 The Extent of Asymptotic Stability- Global Asymptotic Stability}

\textbf {Theorem 5.6} Let there exist a scalar function V(y) such that: \\
(i) V(y) is positive definite on $E_n$ and $V(y) \to \infty$ as $||y|| \to \infty$; \\
(ii) with respect to (5.6), $V*(y) \leq 0$ on $R_n$; \\
(iii) the origin is the only invariant subset of the set $E = \{y|V*(y) = 0\}$ \\
Then the zero solution of v'=f(y) is gloablly asmptotically stable.

\textbf {Corollary 1} Let there exist a scalar function V(y) that satisfies (i) above and that has V*(y) negative definite. Then the zero solution of y'=f(y) is globally asymptotically stable. 

\textbf {Corollary 2} If only (i) and (ii) of Theorem 5.6 are satisfied, then all solutions of y'=f(y) are bounded for $t \geq 0$ (that is, (5.6) is Lagrange stable).

\textbf {Chapter 5.6 Nonautonomous Systems}

\textbf {Definition 1} The scalar function V(t,y) is said to be positive definite on the set $\Omega$ if and only if V(t,0) = 0 and there exists a scalar function W(y) independent of t, with $V(t,y) \geq W(y)$ for (t,y) in $\Omega$ and such that W(y) is positive definite in the sense of Definition 1 (Section 5.2).

\textbf {Definition 2} The scalar function V(t,y) is negative definite on $\Omega$ if and only if -V(t,y) is positive on $\Omega$

\textbf {Definition 3} The derivative of V(t,y) with respect to the system y'=f(t,y) is 
\begin{center}
$V*(t,y) = \frac{\partial V}{\partial t} (t,y) + \sum_{j=1}^n \frac{\partial V}{\partial y_j} (t,y) f_j(t,y)$
\end{center}
If $\phi$ is any solution of (5.41), we have
\begin{center}
$\frac{d}{dt} V(t, \phi(t)) = V*(t, \phi (t))$
\end{center}

\textbf {Theorem 5.7} If there exists a scalar function V(t,y) that is positive definite and for which $V*(t,y) \leq 0$ (that is, the derivative (5.42) with respect to the system (5.41) is nonpositive) on some region $\Omega$ that contains the set $H = \{(t,0) | t \geq 0\}$, then the zero solution of y'=f(t,y) is stable.

\textbf {Definition 4} A scalar function U(t,y) is said to satisfy an infinitesimal upper bound if and only if for every $\epsilon > 0$ there exists a $\delta > 0$ such that
\begin{center}
$|U(t,y)| < \epsilon \tab$ on $\{(t,y)| t \geq 0, |y| \leq \delta\}$
\end{center}

\textbf {Theorem 5.8} If there exists a scalar function V(t,y) that is positive definite, satisfies an infinitesimal upper bound, and for which V*(t,y) is negative definite, then the zero solution y'=f(t,y) is asymptotically stable.

\textbf {Definition} The system \\
y'=f(t,y) \\
is said to be asymptotically autonomous on the set $\Omega$ if and only if (a) $\lim_{t \to \infty} f(t,y) = h(y)$ for $y \in \Omega$ and this convergence is uniform for y in closed bounded subsets of $\Omega$ (b) For every $\epsilon > 0$ and every $y \in \Omega$ there exists a $\delta(\epsilon, y)> 0$ such that $|f(t,x) - f(t,y)| < \epsilon$, whenever $|x-y|< \delta$ for $0 \leq t < \infty$.

\textbf {Theorem 5.9} (See p.214 and compare with theorem 5.5) Suppose the system (5.41) is asymptotically autonomous on some set $\Omega$ in y-space. Suppose f(t,y) is bounded for $0 \leq t < \infty$ whenever y lies in a closed bounded set $Q = \{y| |y| \leq K, K > 0\}$. Suppose there exists a nonnegative scalar function V(t,y) such that $V*(t,y) \leq -W(y)$ where $W(y) \geq 0$ with W(y) = 0 only for $y \in \Omega$ (this defines $\Omega$, that is $\Omega = \{ y | W(y) = 0\}$). Let M be the largest positively invariant subset of $\Omega$ with respect to the limiting autonomous system 
\\
\tab y'= h(y) \\
Then every bounded solution of (5.41) approaches M as $t \to \infty$. In particular if all solutions of (5.41) are bounded, then every solution of (5.41) approaches M.

















\end{document}