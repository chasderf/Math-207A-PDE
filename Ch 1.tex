\documentclass{article}
\usepackage{amsmath}
\newcommand\tab[1][1cm]{\hspace*{#1}}
\begin{document}


\title {Math 207A Ordinary Differential Equations: Ch. 1 Existence and uniqueness of Solutions}

\author{Charlie Seager}

\maketitle

\textbf {Chapter 1.4 Vector-Matrix Notation for Systems} We can put derivatives in a matrix.

\textbf {Chapter 1.5 The Need for a Theory} The task of formulating a mathematical model for the motion of a physical system such as the mass-spring system or the simple pendulum leads to a differential equation; different physical approximations of the same system lead to different models (that is, different differential equations)

\textbf {Chapter 1.6 Existence, Uniqueness and Continuity}

\textbf {Theorem 1.2} Let g, $\partial g/ \partial y$ and $\partial g/ \partial z$ be continous in a given region D. Let $(t_0, y_0, z_0)$ be a given point of D. Then there exists an interval containing $t_0$ and exactly one solution $\phi$, defined on this interval of the differential equations $y"=g(t,y,y')$ that passes through $(t_0, y_0, z_0)$ (that is, the solution $\phi$ satisfies the initial conditions $\phi (t_0) = y_0, \phi '(t_0) = z_0)$. The solution exists for those values of t for which the points $(t, \phi (t), \phi '(t))$ lie in D. Further, the solution $\phi$ is a continous function not only of t, but of $t_0, y_0, z_0$ as well (in fact, of the quadruple ($t, t_0, y_0, z_0))$

\textbf {Theorem 1.3} Let h, $\partial h/ \partial y_1 ,..., \partial h / \partial y_n$ be continuous in a given region D. Let $(t_0, \mathcal{n}_1 ,..., \mathcal{n}_n)$ be a given point of D. Then there exists on interval containing $t_0$ and exactly one solution $\phi$ defined on this interval of the differential equations $y^{(n)} = h(t, y, y' ,..., y^{(n-1)})$ that passes through $(t_0, \mathcal{n}_1 ,..., \mathcal{n}_n)$, [that is, the solution $\phi$ satisfies the initial conditions $\phi(t_0) = \mathcal{n}_1, \phi '(t_0) = \mathcal{n}_2 ,...., \phi^{(n-1)}(t_0) = \mathcal{n}_n$]. The solution exists for those values of t for which the points $(t, \phi(t), \phi '(t) ,..., \phi^{(n-1)}(t))$ lie in D. Further the solution $\phi$ is a continous function of the (n+2) variables $t, t_0, \mathcal{n}_1 ,..., \mathcal{n}_n$.

\textbf {Chapter 1.7 The Gronwall Inequality}

\textbf {Theorem 1.4 (Gronwall Inequality)} Let K be a nonnegative constant and let f and g be continous nonnegative function on some interval $\alpha \leq t \leq \beta$ satisfying the inequality
\begin{center}
$f(t) \leq K + \int_\alpha^t f(s) g(s) ds$
\end{center}
for $\alpha \leq t \leq \beta$. Then
\begin{center}
$f(t) \leq K exp (\int_\alpha^t g(s) ds)$
\end{center}
for $\alpha \leq t \leq \beta$










\end{document}